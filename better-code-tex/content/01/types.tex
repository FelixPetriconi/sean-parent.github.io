The notion of a \textit{type} is a fundamental component of programming. Types are created to represent a class of objects. Types can be very general, such as “integer” or very specific such as “index into the collection of documents.”

\comment{This section is referenced by chapter 3}

\section{Definition of \textit{incomplete type}}

A \textit{type} is defined as:

\begin{itemize}
\item A type is a pattern for storing and modifying objects.
\end{itemize}

And an \textit{object} is defined as:

\begin{itemize}
\item An object is a representation of an entity as a value in memory. {::comment}Add reference to section 1.3 of Elements of Programming{:/comment}
\end{itemize}

The fact that an object exists \textit{in memory} is important. Objects are physical entities and as such are governed by the laws of physics, despite being oft referred to as \textit{virtual}.

The basic building block for modern computers is the transistor. A transistor is a solid-state electronic switch. Prior to the use of transistors (the first transistor based computer was built in 1953 \comment{Citation}) computers were built with vacuum tubes, relays, or gears. \comment{citation} Any device that can serve as a controlled switch can be used to build a computer.

% include figure.md name='transistor' caption='Transistor'
% assign figure-transistor = figure-index

The symbol for a simple transistor is shown in {{figure-reference[figure-transistor]}}. Voltage applied to the base allows current to flow from the collector to the emitter.

% include figure.md name='relay' caption='Relay'
% assign figure-relay = figure-number

\comment{Discussion of relationships here? or in chapter 3?}

\comment{Hyrum’s Law: With a sufficient number of users of an API, it does not matter what you promised in the contract, all observable behaviors of your interface will be depended upon by somebody. — Hyrum Wright, Software Engineer a Google}

\section{Why incomplete types are problematic}

\section{How to avoid incomplete types}

\begin{center}
\includesvg[width=0.5\textwidth]{images/transistor}
\end{center}