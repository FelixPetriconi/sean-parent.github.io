% !TEX root = ../../better-code-print.tex

\lettrine[lhang=0.17] To understand what \textit{better code} is we first need to understand what \textit{good code} is. Students are often taught that good code is code that does what the specification says it should. But such an answer begs the question of what is a good specification? Nearly every experienced developer I've met has a snippet of code filed away that has profound beauty - it likely has no corresponding specification and may not even contain a single comment. So what is good code?

By working through this book and applying the ideas within I hope that you will gain a deeper understanding of what good code is, and in striving to write good code you will write better code.

This book presents a collection of software development goals. The word \textit{goal} was chosen carefully. These are not guidelines or rules, and achieving the goal is not always simple or straightforward. Each goal is phrased such that it is not prescriptive, and often states what not to do, but what \textit{to do} is an open ended challenge. You will find that while trying to apply these goals you will sometimes fail, but through the effort you will gain a deeper understanding of your code and learn to write \textit{better code}.

The examples in this book are primarily in C++, this is for two reasons. The majority of my professional career has been spent writing C++ code. I also believe it is the best mainstream language with which to present the ideas within, but at times I will point out some significant shortcomings. The ideas in this book are not limited to the C++ developer, but apply equally if you are programming in any other language. Where I think there is value in illustrating this point I provide examples in other languages. 

\comment{It may just be JavaScript as the other language, as yet undecided}
